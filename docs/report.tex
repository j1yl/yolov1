\documentclass[conference]{IEEEtran}
\usepackage{amsmath}
\usepackage{graphicx}
\usepackage{listings}
\usepackage{xcolor}
\usepackage{hyperref}
\usepackage{algorithm}
\usepackage{algorithmic}
\usepackage{mathtools}
\usepackage{enumitem}
\usepackage{booktabs}
\usepackage{titlesec}
\usepackage{setspace}

% Configure column spacing
\setlength{\columnsep}{4em}

% Configure title format
\titleformat{\section}
  {\normalfont\large\bfseries}
  {\arabic{section}}
  {1em}
  {}
\titlespacing*{\section}{0pt}{1ex plus 0.5ex minus .2ex}{1ex plus .2ex}

\titleformat{\subsection}
  {\normalfont\bfseries}
  {\arabic{section}.\arabic{subsection}}
  {1em}
  {}
\titlespacing*{\subsection}{0pt}{0.75ex plus 0.5ex minus .2ex}{0.75ex plus .2ex}

% Configure listings
\lstset{
    basicstyle=\small\ttfamily,
    breaklines=true,
    frame=single,
    xleftmargin=1em,
    xrightmargin=1em
}

% Configure itemize and enumerate
\setlist{leftmargin=*}
\setlist[itemize]{label=$\bullet$}
\setlist[enumerate]{label=\arabic*.}

% Custom title command
\renewcommand{\maketitle}{%
  \begin{center}
    {\bfseries\large CS190I: Generative AI, Spring 2025\\
    Programming Assignment 1\\
    \normalsize Joe Lee\\
    \today}
  \end{center}
}

\begin{document}

\maketitle

\section{Algorithm and Dataset Selection}
We implemented YOLOv1 following the architecture in ``You Only Look Once: Unified, Real-Time Object Detection'' \cite{redmon2016you}. Our implementation uses the Pascal VOC dataset with a 7$\times$7 grid, 2 bounding boxes per cell, and 20 classes.

\section{Implementation Details}

\subsection{Enhanced Loss Function}
The total loss function combines four components:

\begin{equation}
L_{\text{total}} = \lambda_{\text{coord}}L_{\text{coord}} + L_{\text{obj}} + \lambda_{\text{noobj}}L_{\text{noobj}} + L_{\text{class}}
\end{equation}

where $\lambda_{\text{coord}} = 5$ and $\lambda_{\text{noobj}} = 0.5$. The coordinate loss uses square root scaling for width and height:

\begin{equation}
\begin{aligned}
L_{\text{coord}} = \sum_{i=0}^{S^2} \sum_{j=0}^{B} \mathbf{1}_{ij}^{\text{obj}} [&(x_i - \hat{x}_i)^2 + (y_i - \hat{y}_i)^2 + \\
&(\sqrt{w_i} - \sqrt{\hat{w}_i})^2 + (\sqrt{h_i} - \sqrt{\hat{h}_i})^2]
\end{aligned}
\end{equation}

The objectness and no-object losses are calculated as:

\begin{equation}
\begin{aligned}
L_{\text{obj}} &= \sum_{i=0}^{S^2} \sum_{j=0}^{B} \mathbf{1}_{ij}^{\text{obj}} (C_i - \hat{C}_i)^2 \\
L_{\text{noobj}} &= \sum_{i=0}^{S^2} \sum_{j=0}^{B} \mathbf{1}_{ij}^{\text{noobj}} (C_i - \hat{C}_i)^2 \\
L_{\text{class}} &= \sum_{i=0}^{S^2} \mathbf{1}_{i}^{\text{obj}} \sum_{c \in \text{classes}} (p_i(c) - \hat{p}_i(c))^2
\end{aligned}
\end{equation}

\subsection{Training Pipeline}
Key optimizations include:
\begin{itemize}
    \item Adam optimizer ($lr = 2e-5$, $\beta_1 = 0.9$, $\beta_2 = 0.999$)
    \item Validation every 9 epochs with mAP:
    \begin{equation}
    \text{mAP} = \frac{1}{|C|} \sum_{c \in C} \text{AP}(c)
    \end{equation}
\end{itemize}

\subsection{Data Processing}
Grid cell assignment and normalization:
\begin{equation}
\begin{aligned}
i &= \lfloor y_{\text{center}} \times S \rfloor \\
j &= \lfloor x_{\text{center}} \times S \rfloor \\
x_{\text{norm}} &= \frac{x_{\text{center}} - j}{S} \\
y_{\text{norm}} &= \frac{y_{\text{center}} - i}{S}
\end{aligned}
\end{equation}

\subsection{Post-processing}
Non-maximum suppression uses IoU:
\begin{equation}
\text{IoU}(box_1, box_2) = \frac{\text{area}(box_1 \cap box_2)}{\text{area}(box_1 \cup box_2)}
\end{equation}

\section{Training Statistics}

\subsection{Progress}
\begin{center}
\begin{tabular}{cccc}
\toprule
Epoch & Loss & mAP & Time/Epoch \\
\midrule
1 & [TBD] & [TBD] & [TBD] \\
15 & [TBD] & [TBD] & [TBD] \\
30 & [TBD] & [TBD] & [TBD] \\
45 & [TBD] & [TBD] & [TBD] \\
60 & [TBD] & [TBD] & [TBD] \\
75 & [TBD] & [TBD] & [TBD] \\
90 & [TBD] & [TBD] & [TBD] \\
105 & [TBD] & [TBD] & [TBD] \\
120 & [TBD] & [TBD] & [TBD] \\
135 & [TBD] & [TBD] & [TBD] \\
\bottomrule
\end{tabular}
\end{center}

\section{Running Instructions}

\subsection{Setup and Training}

\begin{center}
Required packages:
\begin{lstlisting}[language=bash]
torch>=2.7.0
numpy>=2.2.5
matplotlib>=3.10.1
Pillow>=11.2.1
torchvision>=0.22.0
torchmetrics>=1.7.1
\end{lstlisting}

Training configuration:
\begin{lstlisting}[language=python]
LEARNING_RATE = 2e-5
BATCH_SIZE = 80
EPOCHS = 135
NUM_WORKERS = 4
PIN_MEMORY = True
\end{lstlisting}
\end{center}

\subsection{GPU Requirements}
\begin{itemize}
    \item Batch size: [TBD]
    \item Memory: [TBD] GB
    \item Training time: [TBD] hours
\end{itemize}

\subsection{Testing on Custom Images}
To test on custom images:
\begin{lstlisting}[language=bash]
python predict.py --image path/to/image.jpg --model checkpoints/best_model.pth.tar
\end{lstlisting}

\section{Lessons Learned}
Loss function weight tuning proved critical for model performance, while square root scaling helped handle varying object sizes. Grid-based detection required special attention to boundary objects.

\begin{thebibliography}{9}
\bibitem{redmon2016you}
J. Redmon, S. Divvala, R. Girshick, and A. Farhadi,
``You Only Look Once: Unified, Real-Time Object Detection,''
\textit{arXiv:1506.02640}, 2016.
\end{thebibliography}

\end{document} 